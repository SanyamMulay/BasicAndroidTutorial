\documentclass[11pt]{beamer}

\usepackage[utf8]{inputenc}
\usepackage{amsmath}
\usepackage{amsfonts}
\usepackage{amssymb}
\usepackage{graphicx}
\usepackage{listings}
\usepackage{hyperref}

\author{Sanyam, Vinay, Rahul, Akshay @EduTechLabs}
\title{Android Basics}
%\setbeamercovered{transparent} 
%\setbeamertemplate{navigation symbols}{} 
%\logo{} 
%\institute{} 
%\date{} 
%\subject{} 

% selecting theme
\usetheme{Berkeley}

%%configure listings package for java
% Default fixed font does not support bold face
\DeclareFixedFont{\ttb}{T1}{txtt}{bx}{n}{8pt} % for bold
\DeclareFixedFont{\ttm}{T1}{txtt}{m}{n}{8pt}  % for normal


%custom colours
\definecolor{pblue}{rgb}{0.13,0.13,1}
\definecolor{pgreen}{rgb}{0,0.5,0}
\definecolor{pred}{rgb}{0.9,0,0}
\definecolor{pgrey}{rgb}{0.46,0.45,0.48}

%setting up hyperref
\hypersetup{
colorlinks = true,
linkcolor = blue,
urlcolor = blue
}


%configuring listings package to highlight java syntax
\newcommand\javastyle{\lstset{
language=Java,
%basicstyle=\ttm,
otherkeywords={this},             % Add keywords here
%keywordstyle=\ttb\color{deepblue},
emph={MyClass,__init__},          % Custom highlighting
emphstyle=\ttb\color{pred},    % Custom highlighting style
%stringstyle=\color{deepgreen},
commentstyle=\color{pgreen},
keywordstyle=\ttb\color{pblue},
stringstyle=\color{pred},
basicstyle=\ttfamily,
moredelim=[il][\textcolor{pgrey}]{$$}, %%$$ %the double dollar to prevent texmaker from going crazy
moredelim=[is][\textcolor{pgrey}]{\%\%}{\%\%}
%frame=tb,                         % Any extra options here
showstringspaces=false            % 
}}

% java environment
\lstnewenvironment{java}[1][]
{
\javastyle
\lstset{#1}
}
{}

% java for external files
\newcommand\javaexternal[2][]{{
\javastyle
\lstinputlisting[#1]{#2}}}

% java for inline
\newcommand\javainline[1]{{\javastyle\lstinline!#1!}}

%%%%%%%%%%%% begin writing the actual document %%%%%%%%%%%%%%%%%
\begin{document}

\begin{frame}
\titlepage
\end{frame}

%\begin{frame}
%\tableofcontents
%\end{frame}

\section{What can be achieved with Android}

\begin{frame}[containsverbatim]{What can you make with Android ?}

	\begin{flushleft}
		\begin{itemize}			
%		\item Almost anything,
		\item people have made satellites from Android phones !! not kidding (here's the \href{http://www.theverge.com/2013/5/3/4297718/nasa-phonesat-android-nexus-returns-images}{link})
		\item coming back to more earthly things:
		\item ftp server, http serve, barcode scanner, qr code reader, navigator 
		\item of-course there is Whatsapp
		\end{itemize}	
	\end{flushleft}

\huge Your imagination is the limit ! \\

\normalsize Almost 
\end{frame}


\section{Installing the ADT bundle}

\begin{frame}[containsverbatim]{What to download ?}
	\begin{flushleft}
	\begin{itemize}
	
		 \item What is stable ?
		 \item How do you know what will work ?
		 \item Why are we not using studio ?
	\end{itemize}	
	\end{flushleft}
\end{frame}

\section{Getting to know the Eclipse, IDE}

\begin{frame}[containsverbatim]{What is an IDE}
	\begin{flushleft}
	\begin{itemize}
	
		\item IDE ? 
		\item Why use and IDE ?
		\item will I have to learn studio all over again ?

	\end{itemize}	
	\end{flushleft}
\end{frame}

\begin{frame}[containsverbatim]{Importing, Exporting Projects}

\end{frame}

\begin{frame}[containsverbatim]{Starting a new project}
	\begin{flushleft}
	\begin{itemize}
	
		\item next, next, next, Finish install :)
		\item What are all these options ? are they important ?
		\item Boom you have an app !
		\item Hey ! wait a minute, was this magic ? 
		\item No, it was templates :P
		\item Batteries included, philosophy 
	\end{itemize}
	\end{flushleft}

\end{frame}

\begin{frame}[containsverbatim]{Now how do I run this project ?}
	\begin{flushleft}
	\begin{itemize}
		\item setting up virtual devices, emulator, geny motion is dead :(
		\item testing on hardware devices, 7 taps ! yup this is magic
		\item connecting to a device on wifi  
	\end{itemize}
	\end{flushleft}

\end{frame}

\begin{frame}[containsverbatim]{Now how do I run this project ?}
	\begin{flushleft}
	\begin{itemize}
		\item setting up virtual devices, emulator, genymotion is dead :(
		\item testing on hardware devices, 7 taps ! yup this is magic
		\item connecting to a device on wifi 
	\end{itemize}
	\end{flushleft}
\end{frame}


\begin{frame}[containsverbatim]{Useful IDE tools}
	\begin{flushleft}
	\begin{itemize}
		\item Eclipse views, perspectives, gutter, line-numbers
		\item Eclipse shortcuts, F11, Ctrl+F11, Ctrl+shift+o, Ctrl+space etc
		\item re-factoring, We know you will mess-up names
		\item LogCat
		\item Code Completion
		\item auto import dependencies
		\item device manager gui (adb)
		\item toolbar customization
		\item eclipse syntax checking and suggestion
		\item getting syntax hints (javadocs for the functions / methods)
		\item getting code completion suggestions
		\item mostly, eclipse is always right
		\item mostly, we are doing something that does not fit in the usual android framework
	\end{itemize}
	\end{flushleft}
\end{frame}


\begin{frame}[containsverbatim]{Why is it so complex ? - General note}
	\begin{flushleft}
	\begin{itemize}
		\item I am having a panic attack now !!
		\item This is so fishing complex ! 
		\item Is there a reason for this madness ?!
		\item Yes, there is :(
		\item Running android is like running an orchestra !
              (everything has its role, and is an essential component)
		\item just to give you an idea.. let us build a phone from your ARM7 processor !
	\end{itemize}
	\end{flushleft}
\end{frame}

\section{Error Handling}

\begin{frame}[containsverbatim]{Error before output ?!}

	\begin{flushleft}
	\begin{itemize}
		\item its true, its true, Errors speak the truth
                log errors ko yu hi badnam karte hain, dhokha to output deta hai :)
		\item errors are our friends
		\item Log.d $->$ the "print" (prince) of android 
		\item Tags - how to use them, what should a tag tell you ?
		\item LogCat advanced features - filtering output, output levels, 
	\end{itemize}
	\end{flushleft}
\end{frame}

\section{Basic UI}

\begin{frame}[containsverbatim]{XML the custom computer language !}
	\begin{flushleft}
	\begin{itemize}
		\item What do you define in an XML ?
		\item where do you tell android to use this xml ?
		\item What are the basic building blocks
		\item View Class
		\item EditText
		\item TextView
		\item Button
		\item Image 
		\item spinner that does not spin 
		\item a progress bar that is round
	\end{itemize}
	\end{flushleft}
\end{frame}

%Make an app which greets you if you fill you name
% It says don't recognize you if you if name is blank !

\begin{frame}[containsverbatim]{Java !}
	\begin{flushleft}
	\begin{itemize}
		\item now begins the ugly part 
		\item roll up your sleeves ! this is going to get dirty
		\item Activity Class
		\item onClicklistener
		\item toast 
	\end{itemize}
	\end{flushleft}
\end{frame}


\begin{frame}[containsverbatim]{The manifest xml}
	\begin{flushleft}
	\begin{itemize}
		\item Android has a linux base !
		\item very strict about permissions
		\item who can do what ? has to explicitly ask for permissions
		\item this file is like meta data for your project

	\end{itemize}
	\end{flushleft}
\end{frame}


\begin{frame}[containsverbatim]{Now ! We can talk structure }
	\begin{flushleft}
	\begin{itemize}
		\item what is the structure of an android app ?
		\item Audience have you been paying attention
		\item look and feel of the app ?
		\item permissions for the app ?
		\item behaviour of the app ?
		\item what are so many folders doing here ? 
	          src, res and sub-folders
		\item auto generated files
		\item gen, bin, 
		\item Where did this R. come from ?
	\end{itemize}
	\end{flushleft}
\end{frame}

\section{Exception Handling}
%Try to handle null filled in the edit text in the previous app 
% Give a debugging trial run
% A Mock for making mistake and finding out what happened.


\section{Custom Toast Message}

\section{Activity Class - details}
%Life cycle

\section{Multiple Activities}
%Intenet
%Passing paramemters from activity to activity - this is not the only way
%Navigation - backstack, clear backstack
%delibratly stop an activity

\section{Layouts}
%View groups

%Linear - horizontal stacking, vertical stacking

%Relative - anchor, positioning of other elements

%Scrolling

%linear, relative, viewgroup, id, width, height, padding

%linear layout --- using XML
%linear layout --- using JAVA  

\section{Fragments}
%How is a fragment different from activity
%Why use a fragment ?
%
%When to use a fragment over and activity ?
%When to use an activity over a fragment ?

\section{Event Handling}
%- what is an event ?
%- what is a listener ?
%- have you seen a listener before ? (You friendly, alright not so friendly server apache ! runs a listener) which port by the way ?
%
%- java the button click event, had a listener !
%- 

\section{Local Storage}
%One form of local storage - 
%shared preferences
%
%Now this is a singleton ! What the heck is this and what do you want to make it more complicated.


\section{Your powers combine ! Let us make an App}

\end{document}