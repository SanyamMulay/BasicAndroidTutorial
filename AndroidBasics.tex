\documentclass[11pt]{beamer}

\usepackage[utf8]{inputenc}
\usepackage{amsmath}
\usepackage{amsfonts}
\usepackage{amssymb}
\usepackage{graphicx}
\usepackage{listings}
\usepackage{hyperref}

\author{Sanyam, Vinay, Rahul, Akshay @EduTechLabs}
\title{Android Basics}
%\setbeamercovered{transparent} 
%\setbeamertemplate{navigation symbols}{} 
%\logo{} 
%\institute{} 
%\date{} 
%\subject{} 

% selecting theme
\usetheme{Warsaw}

%%configure listings package for java
% Default fixed font does not support bold face
\DeclareFixedFont{\ttb}{T1}{txtt}{bx}{n}{8pt} % for bold
\DeclareFixedFont{\ttm}{T1}{txtt}{m}{n}{8pt}  % for normal


%custom colours
\definecolor{pblue}{rgb}{0.13,0.13,1}
\definecolor{pgreen}{rgb}{0,0.5,0}
\definecolor{pred}{rgb}{0.9,0,0}
\definecolor{pgrey}{rgb}{0.46,0.45,0.48}

%setting up hyperref
\hypersetup{
colorlinks = true,
linkcolor = blue,
urlcolor = blue
}


%configuring listings package to highlight java syntax
\newcommand\javastyle{\lstset{
language=Java,
%basicstyle=\ttm,
otherkeywords={this},             % Add keywords here
%keywordstyle=\ttb\color{deepblue},
emph={MyClass,__init__},          % Custom highlighting
emphstyle=\ttb\color{pred},    % Custom highlighting style
%stringstyle=\color{deepgreen},
commentstyle=\color{pgreen},
keywordstyle=\ttb\color{pblue},
stringstyle=\color{pred},
basicstyle=\ttfamily,
moredelim=[il][\textcolor{pgrey}]{$$}, %%$$ %the double dollar to prevent texmaker from going crazy
moredelim=[is][\textcolor{pgrey}]{\%\%}{\%\%}
%frame=tb,                         % Any extra options here
showstringspaces=false            % 
}}

% java environment
\lstnewenvironment{java}[1][]
{
\javastyle
\lstset{#1}
}
{}

% java for external files
\newcommand\javaexternal[2][]{{
\javastyle
\lstinputlisting[#1]{#2}}}

% java for inline
\newcommand\javainline[1]{{\javastyle\lstinline!#1!}}

%%%%%%%%%%%% begin writing the actual document %%%%%%%%%%%%%%%%%
\begin{document}

\begin{frame}
\titlepage
\end{frame}

%\begin{frame}
%\tableofcontents
%\end{frame}

\section{What can be achieved with Android}

\begin{frame}[containsverbatim]{What can you make with Android ?}

	\begin{flushleft}
		\begin{itemize}			
%		\item Almost anything,
		\item people have made satellites from Android phones !! not kidding (here's the \href{http://www.theverge.com/2013/5/3/4297718/nasa-phonesat-android-nexus-returns-images}{link})
		\item coming back to more earthly things:
		\item ftp server, http serve, barcode scanner, qr code reader, navigator 
		\item of-course there is Whatsapp
		\end{itemize}	
	\end{flushleft}

\huge Your imagination is the limit ! \\

\normalsize Almost 
\end{frame}


\section{Installing the ADT bundle}

\begin{frame}[containsverbatim]{What to download ?}
	\begin{flushleft}
		 - What is stable ?
		 - How do you know what will work ?
		 - Why are we not using studio ?
	\end{flushleft}
\end{frame}

\section{Getting to know the Eclipse, IDE}

\begin{frame}[containsverbatim]{What is an IDE}
- IDE ? 
- Why use and IDE ?
- will I have to learn studio all over again ?

\end{frame}

\begin{frame}[containsverbatim]{Importing, Exporting Projects}

\end{frame}

\begin{frame}[containsverbatim]{Starting a new project}
- next, next, next, Finish install :)

- What are all these options ? are they important ?

- Boom you have an app !

- Hey ! wait a minute, was this magic ? 
- No, it was templates :P
- Batteries included, philosophy 
\end{frame}

\begin{frame}[containsverbatim]{Now how do I run this project ?}

- setting up virtual devices, emulator, geny motion is dead :(
- testing on hardware devices, 7 taps ! yup this is magic
- connecting to a device on wifi 
- 
\end{frame}

\begin{frame}[containsverbatim]{Now how do I run this project ?}

- setting up virtual devices, emulator, geny motion is dead :(
- testing on hardware devices, 7 taps ! yup this is magic
- connecting to a device on wifi 
- 
\end{frame}


\begin{frame}[containsverbatim]{Useful IDE tools}

- Eclipse views, perspectives, gutter, line-numbers
- Eclipse shortcuts, F11, Ctrl+F11, Ctrl+shift+o, Ctrl+space etc
- re-factoring, We know you will mess-up names
- LogCat
- Code Completion
- auto import dependencies
- device manager gui (adb)
- toolbar customization
- eclipse syntax checking and suggestion
- getting syntax hints (javadocs for the functions / methods)
- getting code completion suggestions
- mostly, eclipse is always right
- mostly, we are doing something that does not fit in the usual android framework

\end{frame}


\begin{frame}[containsverbatim]{Why is it so complex ? - General note}

- I am having a panic attack now !!
- This is so fishing complex ! 

- Is there a reason for this madness ?!
- Yes, there is :(
- Running android is like running an orchestra !
(everything has its role, and is an essential component)
- just to give you an idea.. let us build a phone from your ARM7 processor !

\end{frame}

\section{Error Handling}

\begin{frame}[containsverbatim]{Error before output ?!}
- its true, its true, Errors speak the truth
log errors ko yu hi badnam karte hain, dhokha to output deta hai :)
- errors are our friends
- Log.d --> the "print" (prince) of android 
- Tags - how to use them, what should a tag tell you ?
- LogCat advanced features - filtering output, output levels, 

\end{frame}

\section{Basic UI}

\begin{frame}[containsverbatim]{XML the custom computer language !}

- What do you define in an XML ?
- where do you tell android to use this xml ?
- What are the basic building blocks
- View Class
- EditText
- TextView
- Button
- Image 

- spinner that does not spin 
- a progress bar that is round
\end{frame}


%Make an app which greets you if you fill you name
% It says don't recognize you if you if name is blank !

\begin{frame}[containsverbatim]{Java !}

- now begins the ugly part 
- roll up your sleeves ! this is going to get dirty


- Activity Class

- onClicklistener

- toast 
\end{frame}

\begin{frame}[containsverbatim]{The manifest xml}

- Android has a linux base !
- very strict about permissions
- who can do what ? has to explicitly ask for permissions
- this file is like meta data for your project

\end{frame}

\begin{frame}[containsverbatim]{Now ! We can talk structure }

- what is the structure of an android app ?
- Audience have you been paying attention
- look and feel of the app ?
- permissions for the app ?
- behaviour of the app ?
- what are so many folders doing here ? 
	src, res and sub-folders
- auto generated files
- gen, bin, 
- Where did this R. come from ?

\end{frame}


\section{Exception Handling}
%Try to handle null filled in the edit text in the previous app 
% Give a debugging trial run
% A Mock for making mistake and finding out what happened.


\section{Custom Toast Message}

\section{Activity Class - details}
%Life cycle

\section{Multiple Activities}
%Intenet
%Passing paramemters from activity to activity - this is not the only way
%Navigation - backstack, clear backstack
%delibratly stop an activity

\section{Layouts}
%View groups

%Linear - horizontal stacking, vertical stacking

%Relative - anchor, positioning of other elements

%Scrolling

%linear, relative, viewgroup, id, width, height, padding

%linear layout --- using XML
%linear layout --- using JAVA  

\section{Fragments}
%How is a fragment different from activity
%Why use a fragment ?
%
%When to use a fragment over and activity ?
%When to use an activity over a fragment ?

\section{Event Handling}
%- what is an event ?
%- what is a listener ?
%- have you seen a listener before ? (You friendly, alright not so friendly server apache ! runs a listener) which port by the way ?
%
%- java the button click event, had a listener !
%- 

\section{Local Storage}
%One form of local storage - 
%shared preferences
%
%Now this is a singleton ! What the heck is this and what do you want to make it more complicated.


\section{Your powers combine ! Let us make an App}

\end{document}