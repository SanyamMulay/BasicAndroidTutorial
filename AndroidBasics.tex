\documentclass[11pt]{beamer}

\usepackage[utf8]{inputenc}
\usepackage{amsmath}
\usepackage{amsfonts}
\usepackage{amssymb}
\usepackage{graphicx}
\usepackage{listings}

\author{Sanyam, Vinay, Rahul, Akshay @EduTechLabs}
\title{Android Basics}
%\setbeamercovered{transparent} 
%\setbeamertemplate{navigation symbols}{} 
%\logo{} 
%\institute{} 
%\date{} 
%\subject{} 

% selecting theme
\usetheme{Warsaw}

%%configure listings package for java
% Default fixed font does not support bold face
\DeclareFixedFont{\ttb}{T1}{txtt}{bx}{n}{8pt} % for bold
\DeclareFixedFont{\ttm}{T1}{txtt}{m}{n}{8pt}  % for normal

%custom colours
\definecolor{pblue}{rgb}{0.13,0.13,1}
\definecolor{pgreen}{rgb}{0,0.5,0}
\definecolor{pred}{rgb}{0.9,0,0}
\definecolor{pgrey}{rgb}{0.46,0.45,0.48}

%configuring listings package to highlight java syntax
\newcommand\javastyle{\lstset{
language=Java,
%basicstyle=\ttm,
otherkeywords={this},             % Add keywords here
%keywordstyle=\ttb\color{deepblue},
emph={MyClass,__init__},          % Custom highlighting
emphstyle=\ttb\color{pred},    % Custom highlighting style
%stringstyle=\color{deepgreen},
commentstyle=\color{pgreen},
keywordstyle=\ttb\color{pblue},
stringstyle=\color{pred},
basicstyle=\ttfamily,
moredelim=[il][\textcolor{pgrey}]{$$}, %%$$ %the double dollar to prevent texmaker from going crazy
moredelim=[is][\textcolor{pgrey}]{\%\%}{\%\%}
%frame=tb,                         % Any extra options here
showstringspaces=false            % 
}}

% java environment
\lstnewenvironment{java}[1][]
{
\javastyle
\lstset{#1}
}
{}

% java for external files
\newcommand\javaexternal[2][]{{
\javastyle
\lstinputlisting[#1]{#2}}}

% java for inline
\newcommand\javainline[1]{{\javastyle\lstinline!#1!}}

%%%%%%%%%%%% begin writing the actual document %%%%%%%%%%%%%%%%%
\begin{document}

\begin{frame}
\titlepage
\end{frame}

%\begin{frame}
%\tableofcontents
%\end{frame}

\section{What can be achieved with Android}

\begin{frame}[containsverbatim]{What can you make with Android ?}
- Almost anything,
- people have made satellites from Android phones !! not kidding (here's the link)
- coming back to more earthly things:
- ftp server, http serve, barcode scanner, qr code reader, navigator 
- of-course there is Whatsapp

\end{frame}


\section{Installing the ADT bundle}

\begin{frame}[containsverbatim]{What to download ?}
 - What is stable ?
 - How do you know what will work ?
 - Why are we not using studio ?
\end{frame}

\section{Getting to know the Eclipse, IDE}

\begin{frame}[containsverbatim]{What is an IDE}
- IDE ? 
- Why use and IDE ?
- will I have to learn studio all over again ?

\end{frame}

\begin{frame}[containsverbatim]{Importing, Exporting Projects}

\end{frame}

\begin{frame}[containsverbatim]{Starting a new project}
- next, next, next, Finish install :)

- What are all these options ? are they important ?

- Boom you have an app !

- Hey ! wait a minute, was this magic ? 
- No, it was templates :P
- Batteries included, philosophy 
\end{frame}

\begin{frame}[containsverbatim]{Now how do I run this project ?}

- setting up virtual devices, emulator, geny motion is dead :(
- testing on hardware devices, 7 taps ! yup this is magic
- connecting to a device on wifi 
- 
\end{frame}

\begin{frame}[containsverbatim]{Now how do I run this project ?}

- setting up virtual devices, emulator, geny motion is dead :(
- testing on hardware devices, 7 taps ! yup this is magic
- connecting to a device on wifi 
- 
\end{frame}


\begin{frame}[containsverbatim]{Useful IDE tools}

- Eclipse views, perspectives
- Eclipse shortcuts, F11, Ctrl+F11, Ctrl+shift+o, Ctrl+space etc
- 

\end{frame}



\end{document}